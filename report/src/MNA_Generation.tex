\section{Modified Nodal Analysis (MNA)}
Modified Nodal Analysis (MNA) is the systematic procedure for generating the appropriate simulation matrices. In this section, some element stamps and the notation used in both report and code for the various MNA matrices are introduced.
\subsection{MNA Stamps}
\turmeric has two classes of elements:
\begin{itemize}
    \item Current defined
    \item Voltage defined
\end{itemize}

Current defined elements include resistors, capacitors, and independent current sources. 

\subsection{MNA Generation}
After \turmeric parses the netlist, a special method in the circuit class ().


\subsection{The Circuit Simulation Problem}
In any given analysis, there are many terms that \turmeric needs to consider.

\begin{equation}
    \mathbf{D}\cdot \frac{d\colvec{x}}{dt} + \mathbf{M}\cdot \colvec{x} + \colvec{Z}\left( t\right) + \colvec{N}\left(\colvec{x}\right) = 0 \label{eq:problem}
\end{equation}

\begin{itemize}
    \item $\mathbf{M}$ is the reduced MNA matrix. $\mathbf{M_0}$ refers to the unreduced form generated by the circuit class
    \item $\colvec{x}$ is the vector of node voltages and auxiliary branch currents for voltage defined elements; the system solution
    \item $\colvec{Z}\left( t\right)$ is composed of DC and AC components of input signals. \[ \colvec{Z}\left(t\right) = \colvec{Z}_{DC} + \colvec{Z}_{AC}\left(t \right)\]
    \item MNA treats non-linear elements as current sources. These elements are non-linear with respect to the state variables. $\colvec{N}(\colvec{x})$ denotes the addition of non-linear circuit elements
    \item \textit{Turmeric} also needs to consider the contribution \textit{dynamic} components; capacitors and inductors. The contribution of these elements is stored in the $\mathbf{D}$ matrix. $\frac{d\colvec{x}}{dt}$ is the time derivative of the system.
\end{itemize}

The full problem can be expanded as follows:

\begin{align}
    &\mathbf{D}\cdot \frac{d\colvec{x}}{dt} + \mathbf{M}\cdot \colvec{x} + \colvec{Z}_{DC} + \colvec{Z}_{AC}\left( t\right) + \colvec{NL}\left(\colvec{x}\right) = 0 \label{eq:full}
\end{align}