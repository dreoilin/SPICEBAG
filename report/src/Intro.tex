\section{Introduction}
Probably the most famous circuit simulator is SPICE (Simulation Program with Integrated Circuit Emphasis). SPICE1 was developed in University of California, Berkeley in 1973, and \\

Our goal was simple: emulate some of the basic functionality of a SPICE simulator in a modern codebase. The source code for SPICE1, SPICE2, and SPICE3 is freely available online, but are difficult to understand for a modern audience unfamiliar with languages like FORTRAN or C. Furthermore, tasks like netlist parsing (which include large amounts of string processing) and waveform display are objectively more suited to a modern language like python. The f2py package makes interfacing between python and and pre-compiled FORTRAN modules exceptionally easy.\\

We decided to implement:
\begin{itemize}
    \item \textbf{a parser} in python to parse a SPICE netlist and populate a ``network" object
    \item \textbf{core numerical routines} in pre-compiled FORTRAN subroutines
    \item \textbf{circuit element models} in python classes
\end{itemize}

The report is structured in ########## sections. A #### and final section analyses the performance of the simulator. All programming, documentation, and accompanying presentation was developed as a team.