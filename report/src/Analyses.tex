\section{Analyses}
The notation introduced in this section is consistent with variable names used in the \turmeric source code.

Turmeric provides four modes of operation:

\begin{itemize}
    \item Operating point analysis
    \item DC analysis
    \item Transient analysis
    \item AC analysis
\end{itemize}

\subsection{The Circuit Simulation Problem}
A non-linear transient simulation is the most complex task that \textit{Turmeric} solves.

\begin{equation}
    \mathbf{D}\cdot \frac{d\colvec{x}}{dt} + \mathbf{M}\cdot \colvec{x} + \colvec{Z}\left( t\right) + \colvec{N}\left(\colvec{x}\right) = 0 \label{eq:problem}
\end{equation}

\begin{itemize}
    \item $\mathbf{M}$ is the reduced MNA matrix. $\mathbf{M_0}$ refers to the unreduced form generated by the circuit class
    \item $\colvec{x}$ is the vector of node voltages and auxiliary branch currents for voltage defined elements; the system solution
    \item $\colvec{Z}\left( t\right)$ is composed of DC and AC components of input signals. \[ \colvec{Z}\left(t\right) = \colvec{Z}_{DC} + \colvec{Z}_{AC}\left(t \right)\]
    \item MNA treats non-linear elements as current sources. These elements are non-linear with respect to the state variables. $\colvec{N}(\colvec{x})$ denotes the addition of non-linear circuit elements
    \item \textit{Turmeric} also needs to consider the contribution \textit{dynamic} components; capacitors and inductors. The contribution of these elements is stored in the $\mathbf{D}$ matrix. $\frac{d\colvec{x}}{dt}$ is the time derivative of the system.
\end{itemize}

The full problem can be expanded as follows:

\begin{align}
    &\mathbf{D}\cdot \frac{d\colvec{x}}{dt} + \mathbf{M}\cdot \colvec{x} + \colvec{Z}_{DC} + \colvec{Z}_{AC}\left( t\right) + \colvec{N}\left(\colvec{x}\right) = 0 \label{eq:full}
\end{align}

\subsection{Operating Point Analysis}
The fundamental operation of the simulator is an operating point analysis. An operating point solution is used as an initial estimate in a transient simulation. When calculating the DC operating point, some adjustments can be made to \cref{eq:full}:

\begin{itemize}
    \item Capacitors are considered open circuits and inductors short circuits. As such they have no impact on the $\mathbf{M}$ matrix. Inductors behave as a short circuit \footnote{in the MNA equations, this is a 0 V voltage source}
    \item Furthermore, the system is static. There is no time derivative and as such contributions from the dynamic matrix are zero \[\mathbf{D} = [0] \]
    \item Time-dependent voltage and current sources are open and short circuits respectively. Only DC contributions are considered \[ \colvec{Z}_{AC}\left( t\right) = \colvec{0}\]
\end{itemize}

The operating point problem reduces to:

\begin{equation}
    \mathbf{M}\cdot \colvec{x} + \colvec{Z}_{DC} + \colvec{N}\left(\colvec{x} \right) = 0 \label{eq:op problem}
\end{equation}

\subsubsection{The Linear Case}
If the circuit only consists of linear elements, the problem can be solved by means of a linear systems solver. \turmeric uses an LU decomposition method. The solver was implemented in FORTRAN and compiled using \texttt{f2py}.

\begin{listing*}[!p]
\inputminted[firstline=1, lastline=1, linenos=true]{octave}{lst/newton.m}
\inputminted[firstline=33, lastline=86, linenos=true]{octave}{lst/newton.m}
\caption{Implementation of Newton-Raphson to solve a system of equations - \textit{newton.m}}
\label{listing:newrap code}
\end{listing*}

\subsubsection{The Non-Linear Case}
If a circuit contains a diode or a CMOS transistor, there is a little more work to be done.\\

\begin{enumerate}
    \item[1.] The $M$ matrix is augmented by adding a small conductance ($G_{min}$) from each node to ground to aid convergence. A user can specify this conductance, but it defaults to $\SI{1d-12}{\siemens}$. The $G_{min}$ matrix ($\mathbf{G}$) is a zero matrix whose diagonal entries are the $G_{min}$. Mathematically, \[ \mathbf{G} = G_{min}\mathbf{I}\]
    
    THIS IS NOT THIS SIMPLE
    \item[2.] The system (after this operation) looks like this: \[ \left(M + G \right)\cdot \colvec{x} + \colvec{Z}_{DC} +  \colvec{N}\left(\colvec{x} \right) = 0\]
    This is an implicit equation in $\colvec{x}$ and can only be solved wth an iteration method. \turmeric employs a damped newton method with user modifiable maximum iteration and tolerance values. 
\end{enumerate}




\subsubsection{MNA Generation}
$Z\left( t\right)$ is typically made up of DC and AC components. Turmeric distinguishes between these sources; $Z_{DC}$ and $Z_{AC}\left( t\right)$. The MNA generation creates the unreduced $M$ and $Z_{DC}$ matrices.